\documentclass[a4paper, 12pt]{scrartcl} % Font size (can be 10pt, 11pt or 12pt) and paper size (remove a4paper for US letter paper)
\input{include/header}
\usepackage{lmodern}
\usepackage{hyperref}
\usepackage{longtable}
\usepackage{abbrevs}
\usepackage[toc]{glossaries} % nomain, if you define glossaries in a file, and you use \include{INP-00-glossary}
\makeglossaries
\linespread{1.05} % Change line spacing here, Palatino benefits from a slight increase by default
\makeatletter
\renewcommand{\maketitle}{ % Customize the title - do not edit title and author name here, see the TITLE block below
  \begin{flushright} % Right align
    {\LARGE\@title}\\ % Increase the font size of the title
    \vspace{50pt} % Some vertical space between the title and author name
    {\large\@author} % Author name
    \\\@date % Date
    \vspace{40pt} % Some vertical space between the author block and abstract
  \end{flushright}
}
\title{\textbf{INX 365 Assignment 2}\\ % Title
Distributed Communication} % Subtitle
\author{\textsc{Lukas Elsner \\ n9224262} % Author
\\{\textit{Queensland University of Technology}}} % Institution
\date{\today} % Date
\begin{document}
\maketitle % Print the title section
\thispagestyle{empty}
%\renewcommand{\abstractname}{Summary} % Uncomment to change the name of the abstract to something else
\vspace{10em}
\begin{abstract}
The task was to develop a client/server application for professional dietitians
using C and BSD sockets. The server should load a csv-file with food and their
nutrition values. Furthermore, it must handle multiple client sessions at once.
Every client can query food by search terms and add new food to the list. This
document covers the implementation report, as well as occurred problems while
implementing the application.
\end{abstract}
\newpage
\tableofcontents
\newpage
%\hspace*{3,6mm}\textit{Keywords:} lorem , ipsum , dolor , sit amet , lectus % Keywords
%\vspace{30pt} % Some vertical space between the abstract and first section
%----------------------------------------------------------------------------------------
%	ESSAY BODY
%----------------------------------------------------------------------------------------

\section{Statement of Completeness}

All three tasks has been implemented as requested. Some minor additional changes
were made. This report will cover the implementation details, as well as the
additional changes.

\subsection{Basic Functionality}

The following basic functionality was implemented:
\begin{itemize}
    \item Server automatically loads calories.csv from the current path. If the
      csv-file is in the same directory as the server binary, it must be run
      from this directory.
    \item The server's listening port can be changed with help of a command
      line parameter
    \item The server initiates a clean shutdown when it receives the SIGINT
      signal.
    \item The server can handle 10 client connections in parallel. A
      producer/consumer pattern was used, as Task 2 suggested this.
    \item Food can be added from a client by pressing `a' and then following
      the on screen instructions to supply the needed attributes.
    \item The client accepts two command line parameters. The server's
      IP-address and its port.
    \item When the user enters a food name, the server processes the request
      and sends the search results back to the client.
    \item If a search query results in no findings, an appropriate error message
      is printed on the client side.
    \item The client can be shut down by pressing `q'
  \end{itemize}

In addition to that, the following extras were implemented:
\begin{itemize}
  \item Running the client or server with $-h$ parameter will print the usage help.
    \item The client connects to 127.0.0.1 on port 12345 when started without parameters.
    \item The client connects to 127.0.0.1 on given port with only one parameter.
\end{itemize}

\subsection{Process Synchronisation and Coordination}

\subsubsection{Connection handling}
To be able to handle multiple connections, a thread pool is created. Ten threads are consuming the clients produced by the server thread. A consumer/producer pattern has been used with help of mutexed, semaphores and a rotating array for the data which has to be accessed by multiple threads.

\subsubsection{Data handling}
All food is stored in a linked list. Multiple threads can read the list at any time, but only one thread can write. This is implemented by the Reader/Writer pattern from lecture 5.

\subsection{Non-Functionality}
Valgrind confirmed that there are no possible memory leaks, even
after a long runtime of both the client and the server.\\
Also, the code is well commented and a doxygen reference manual was created. \\
There are no known issues regarding to unexpected behavior, such as crashes.

%----------------------------------------------------------------------------------------
%	BIBLIOGRAPHY
%----------------------------------------------------------------------------------------
%\bibliographystyle{IEEEtran}
%\bibliography{bibtex}
%----------------------------------------------------------------------------------------
%\printindex
%\printglossaries
%\listoffigures
%\listoftables
\end{document}
